% La classe standard pour faire des slides en \LaTeX est la classe « beamer » :
%%%%%%%%%%%%%%%%%%%%%%%%%%%%%%%%%%%%%%%%%%
\documentclass[12pt]{beamer}
%%%%%%%%%%%%%%%%%%%%%%%%%%%%%%%%%%%%%%%%%%
% Comme pour un fichier tex normal, on déclare les packages dans le préambule (i.e. ici, avant le « begin{document} ») :
\usepackage[utf8]{inputenc}
\usepackage[T1]{fontenc}
\usepackage[french]{babel}
\usepackage{url}

\usepackage{times}
\usepackage{tikz}

\usepackage{verbatim}
\usetikzlibrary{arrows,shapes}

% On définit le style des transparents par la commande \usetheme :
\usetheme{Darmstadt}
% Il existe déjà de nombreaux autres styles en beamer : Juan-Les-Pins, Berlin,...
% On peut également se passer de style.

%%
\title[Faire une présentation, en LaTex]{Faire une présentation, en \LaTeX}
\author[M. Senot]{Maxime Senot\footnote{À partir du document de Lucie Martinet \& Damien Stehlé}}
\institute{Lycée EIB \'Etoile -- Paris\\
% 1\textsuperscript{ère} --- Projet de programmation}
T\textsuperscript{ale} --- Projet de programmation%
}

% \date[17/11/2020]{17 novembre 2020}


% Cette commande permet d'afficher le plan a chaque changement de section
%%%%%
%\AtBeginSection[]
%{
%  \begin{frame}<beamer>
 %   \frametitle{Plan}
 %   \tableofcontents[currentsection]
 % \end{frame}
%}
%%%%%%
% On peut personnaliser les en-têtes et pieds de slides :
\defbeamertemplate*{footline}{infolines theme}
{
 \hbox{%
  \begin{beamercolorbox}[wd=.22\paperwidth,ht=2.25ex,dp=1ex,center]{author in head/foot}%
    \insertshortauthor%
  \end{beamercolorbox}%
  \begin{beamercolorbox}[wd=.57\paperwidth,ht=2.25ex,dp=1ex,center]{title in head/foot}%
    \insertshorttitle%
  \end{beamercolorbox}%
  \begin{beamercolorbox}[wd=.13\paperwidth,ht=2.25ex,dp=1ex,center]{date in head/foot}%
    \insertshortdate{}%
  \end{beamercolorbox}%
  \begin{beamercolorbox}[wd=.08\paperwidth,ht=2.25ex,dp=1ex,center]{author in head/foot}%
    \insertframenumber/\inserttotalframenumber%
  \end{beamercolorbox}}%
}

%%%%%%%%%%%%%%%%%%%%%%%%%%%%%%%%%%%%%%%%%%
\begin{document}
%%%%%%%%%%%%%%%%%%%%%%%%%%%%%%%%%%%%%%%%%%

\begin{frame}
  \titlepage
\end{frame}

%%%%

\begin{frame}{Introduction}
  \begin{exampleblock}{Objectifs de cette présentation}
  \begin{itemize}
    \item Des conseils pour faire un bon exposé scientifique
    \item Comment utiliser \LaTeX pour faire un exposé
  \end{itemize}
  \end{exampleblock}
\end{frame}

%%%%

\begin{frame}
  \frametitle{Plan}
  \tableofcontents
\end{frame}

%%%%%%%%%%%%%%%%%%%%%%%%%%%%%%%%%%%%%%%%%%
\section{Quel contenu?}
%%%%%%%%%%%%%%%%%%%%%%%%%%%%%%%%%%%%%%%%%%

\begin{frame}{Que raconter?}
  Déterminer les informations que l'on veut faire passer
  \begin{itemize}
    \item Pas trop de détails
    \item Dégager les grandes idées, les points importants 
    \item Les présenter logiquement
  \end{itemize}

  \bigskip

  \uncover<2->{
    Le discours doit être \textcolor{red}{structuré}
    \begin{itemize}
      \item Une introduction
      \item Des parties
      \item Une conclusion 
    \end{itemize}
  }
\end{frame}

%%%%

\begin{frame}{Le niveau technique}
  \vspace*{-.2cm}
  Connaître son public, et s'y adapter:
  \begin{itemize}
    \item Quelle est son niveau de connaissance?
    \item Quel est son niveau technique?
    \item Ne pas sur-estimer son public!
  \end{itemize}

  \medskip
  \uncover<2->{
    \begin{alertblock}{Dans le cas d'une présentation d'un rapport}
      Les examinateurs ne connaissent pas les détails de votre rapport sur le bout des doigts!
      Ni même leur sujet, parfois!
    \end{alertblock}
  }
  \uncover<3->{
    \medskip
    Public hétérogène: tous les membres de l'auditoire doivent pouvoir comprendre au moins un peu.
  }
\end{frame}

%%%%%%%%%%%%%%%%%%%%%%%%%%%%%%%%%%%%%%%%%%
\section{Comment passer un message}
%%%%%%%%%%%%%%%%%%%%%%%%%%%%%%%%%%%%%%%%%%

\begin{frame}{Un exposé n'est pas un rapport}
  \begin{alertblock}{}
    Passer son rapport au vidéoprojecteur n'est pas une option\ldots
  \end{alertblock}

  \medskip

  Des transparents \textcolor{red}{clairs}:
  \begin{enumerate}
    \item<2-> Une grosse taille de caractères, et une police neutre
    \item<3-> Un minimum d'abréviations
    \item<4-> \'Eviter les cassures de lignes inélégantes
    \item<5-> Ne pas souligner
    \item<6-> Limiter les équations (en nombre et en complexité) \\  et les algorithmes
  \end{enumerate}
\end{frame}

%%%%

\begin{frame}{Exploiter l'aspect visuel}
  \begin{itemize}
    \item Des images et des dessins 
    \item Des tableaux, graphiques, histogrammes (mais simples!)
    \item Utiliser les couleurs mais:
      \begin{itemize}
	\item pas de vert (ou jaune)
	\item pas des couleurs claires 
	\item pas d'effet arc-en-ciel
      \end{itemize}
  \end{itemize}

  \uncover<3->{
    \begin{alertblock}{Et les animations?}
      Éviter d'en faire trop et utiliser si cela permet d'expliquer particulièrement bien (généralement dans les dessins).
    \end{alertblock}
  }
\end{frame}

%%%%%%%%%%%%%%%%%%%%%%%%%%%%%%%%%%%%%%%%%%
\section{Questions pratiques}
%%%%%%%%%%%%%%%%%%%%%%%%%%%%%%%%%%%%%%%%%%

\begin{frame}{Combien?}
  \begin{alertblock}{Le principe général}
    Ne pas gaver son public... surtout s'il enchaîne les exposés
  \end{alertblock}

  \medskip

  \uncover<2->{
    Quelques ordres de grandeur:
    \begin{itemize}
      \item $\gg$ 45 sec par transparent, voire plutôt $\gg$ 60 sec \\ (à adapter aux circonstances)
      \item Une idée principale par transparent 
      \item Si plus de 10-15 transparents, mettre un transparent de plan entre les parties, et respirer un coup
      \item Numéroter ses transparents
    \end{itemize}
  }
\end{frame}

%%%%

\begin{frame}{Comment gérer le déroulement à l'oral}
  \begin{block}{D\'emarrer\ldots}
    Avec au moins une phrase introductive, par exemple sur le contexte de l'exposé
  \end{block}

  \uncover<2->{
    \begin{block}{Terminer...}
      En faisant un résumé et en élargissant le spectre.\\
      L'auditoire doit comprendre que vous avez fini de parler.
    \end{block}
  }

  \uncover<3->{
    \begin{block}{Enchaîner...}
      Annoncer un plan au début {\footnotesize (avec un transparent, ou à l'oral si l'exposé est très court)}, 
      et le  rappeler {\footnotesize (au moins  à l'oral)} quand \\ 
      on change de partie.
    \end{block}
  }

\end{frame}

%%%%%%%%%%%%%%%%%%%%%%%%%%%%%%%%%%%%%%%%%%
\section{Autour des transparents}
%%%%%%%%%%%%%%%%%%%%%%%%%%%%%%%%%%%%%%%%%%

\begin{frame}{La tenue... des remarques de bon sens...}
  \begin{itemize}
    \item Se tenir debout
    \item<2-> Avoir une apparence neutre: \\
    c'est le contenu de l'exposé qui doit attirer l'attention
    \item<3-> \^Etre réveillé, mais pas sur-excité
    \item<4-> Ne pas parler trop vite, et ne pas avaler les syllabes
    \item<5-> Regarder son auditoire: c'est \`a lui que vous parlez!
  \end{itemize}
\end{frame}

%%%%

\begin{frame}{Se préparer}
  Répéter, mais pas trop: 
  \begin{itemize}
    \item connaître le contenu, mais sans le connaître par cœur
    \item se chronométrer
  \end{itemize}

  \medskip

  \begin{alertblock}{Et surtout}
    Toujours avoir au moins \textcolor{red}{\bf deux} solutions techniques.
    \begin{itemize}
      \item portable perso 
      \item clé USB
      \item fichier {\tt .pdf} envoyé auparavant aux organisateurs
    \end{itemize}
  \end{alertblock}
\end{frame}

%%%%%%%%%%%%%%%%%%%%%%%%%%%%%%%%%%%%%%%%%%
\section{Beamer}
%%%%%%%%%%%%%%%%%%%%%%%%%%%%%%%%%%%%%%%%%%

\begin{frame}{Beamer}
  Beamer, qu'est-ce?
  \begin{itemize}
    \item Un package \LaTeX~pour faire de beaux transparents.
    \item L'option par défaut pour un exposé scientifique, \\ au moins en informatique
  \end{itemize}
\end{frame}

\begin{frame}{Exemple de code Beamer}
\begin{lstlisting}
 
\end{lstlisting}

\footnotesize
\tt{
$\backslash$begin\{frame\}\{Mon titre\}

du texte
  
$\backslash$uncover<2-> \{

$\backslash$begin\{block\}\{Définition\}

Un code est dit $\backslash$textbf\{bien écrit\} s'il est correct (syntaxiquement et sémantiquement), bien structuré, bien commenté et avec des algorithmes simples à comprendre.$\backslash\backslash$

$\backslash$uncover<3->\{Le code est dit $\backslash$textbf\{cool\} s'il est bien écrit et contient des commentaires marrants et des easter eggs.\}

$\backslash$end\{block\}

\}

encore du texte

$\backslash$end\{frame\}
}
\end{frame}

\begin{frame}{Mon titre}

  du texte

\uncover<2->{
\begin{block}{Définition}
Un code est dit \textbf{bien écrit} s'il est correct (syntaxiquement et sémantiquement), bien structuré, bien commenté et avec des algorithmes simples à comprendre.\\ \uncover<3->{Le code est dit \textbf{cool} s'il est bien écrit et contient des commentaires marrants et des easter eggs.}
\end{block}
}

encore du texte
\end{frame}

\begin{frame}{Dessins et animations}
\begin{itemize}
\item {\tt $\backslash$includegraphics} est l'option basique mais les dessins sont statiques
\item l'utilisation des langages comme {\tt PGF/TikZ} est une des alternatives
\end{itemize}

\bigskip

\uncover<2->{
les avantages des dessins programmés en \LaTeX :

\begin{itemize}
\item des vraies animations avec {\tt $\backslash$uncover, $\backslash$only}
\item un seul fichier source
\item pas besoin de faire la distinction entre les différents formats de dessin
 pour choisir le mode de compilation : {\tt latex} ou {\tt pdflatex}
\end{itemize}
}
\end{frame}


\begin{frame}
\frametitle{Exemple - The \TeX\ work flow}


\tikzstyle{format} = [draw, thin, fill=blue!20]
\tikzstyle{medium} = [ellipse, draw, thin, fill=green!20, minimum height=2.5em]

\begin{figure}
\begin{tikzpicture}[node distance=3cm, auto,>=latex', thick]
    % We need to set at bounding box first. Otherwise the diagram
    % will change position for each frame.
    \path[use as bounding box] (-1,0) rectangle (10,-2);
    \path[->]<1-> node[format] (tex) {.tex file};
    \path[->]<2-> node[format, right of=tex] (dvi) {.dvi file}
                  (tex) edge node {\TeX} (dvi);
    \path[->]<3-> node[format, right of=dvi] (ps) {.ps file}
                  node[medium, below of=dvi] (screen) {screen}
                  (dvi) edge node {dvips} (ps)
                        edge node[swap] {xdvi} (screen);
    \path[->]<4-> node[format, right of=ps] (pdf) {.pdf file}
                  node[medium, below of=ps] (print) {printer}
                  (ps) edge node {ps2pdf} (pdf)
                       edge node[swap] {gs} (screen)
                       edge (print);
    \path[->]<5-> (pdf) edge (screen)
                        edge (print);
    \path[->, draw]<6-> (tex) -- +(0,1) -| node[near start] {pdf\TeX} (pdf);
\end{tikzpicture}
\end{figure}
\end{frame}


\begin{frame}{Beamer}

  Ressources:
    \begin{itemize}
      \item Petite introduction:\\ \url{http://www.math-linux.com/spip.php?article77}
      \item Une liste de  thèmes:\\ \url{http://www.hartwork.org/beamer-theme-matrix/}
      \item Guide d'utilisateur:\\ \url{http://tug.ctan.org/macros/latex/contrib/beamer/doc/beameruserguide.pdf}
    \end{itemize}

\end{frame}

%%%%

% \begin{frame}{Où trouver cette présentation}
%   \url{http://perso.ens-lyon.fr/maxime.senot/Projet13/PremierBeamer.tar.gz}
% \end{frame}

%%%%%%%%%%%%%%%%%%%%%%%%%%%%%%%%%%%%%%%%%%
\end{document}
%%%%%%%%%%%%%%%%%%%%%%%%%%%%%%%%%%%%%%%%%%
