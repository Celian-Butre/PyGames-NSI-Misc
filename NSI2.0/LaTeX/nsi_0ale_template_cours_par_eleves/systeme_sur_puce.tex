%%%%%%%%%%%%%%%%%%%%%%%%%%%%%%%%%%%%%%%%%%
\documentclass[a4paper,11pt]{article}
%%%%%%%%%%%%%%%%%%%%%%%%%%%%%%%%%%%%%%%%%%

%% Gestion de l'encodage et de la typographie
\usepackage[utf8]{inputenc}    %% encodage des caractères (en entrée, dans la source .tex) : indispensable pour saisir le code source directement avec des accents (sinon, il faut utiliser e.g. \'e pour é, etc)
\usepackage[T1]{fontenc}       %% gestion de l'encodage (notamment des césure au niveau des caractères accentués)
\usepackage[francais]{babel}   %% traitement du texte adapté aux règles typographiques de la langue donnée en option (e.g., pour l'espacement après les ponctuations)
%% Il existe beaucoup de police de caractères à disposition :                               
% \usepackage{times}
% \usepackage{ae}
\usepackage{lmodern}          %% celle-ci est fortement recommandée
%
\usepackage{graphicx}                  %% pour inclure des figures 
\usepackage{amsmath, amsthm, amssymb}  %% pour faire de jolies mathématiques, avec plein de symboles supplémentaires
\usepackage{url}                       %% pour inclure des urls (hyperref peut egalement servir, si on veut un lien apparent different du "vrai" lien)
\usepackage{hyperref}                  %% pour les liens (à la fois internes et externes au document)
\usepackage[french,ruled,vlined]{algorithm2e}% pour présenter de jolis algos (il existe aussi le package algorithmics)
%
\usepackage[left=1.5cm,right=1.5cm,top=2cm,bottom=2cm,headsep=1cm]{geometry}% pour bidouiller les marges
\usepackage{fancyhdr}         %% personnalisation en-têtes et pieds de page
\usepackage{lastpage}         %% permet de connaître le numéro de la dernière page
\fancyhf{}
%% header personnalisé : left-center-right
\lhead{}\chead{}\rhead{}    %% header personnalisé : left-center-right
%% footer personnalisé : left-center-right
\lfoot{}\cfoot{\sc Système sur puce}\rfoot{\thepage}
\renewcommand{\headrulewidth}{0pt} %% epaisseur trait header
\renewcommand{\footrulewidth}{1pt} %% epaisseur trait footer
\pagestyle{fancy}   %% pour utiliser le style fancy que l'on vient de définir
%
%% QUELQUES REMARQUES ET COMMANDES : 
%% en Tex, on utilise les balises de typo en faisant (comme chez les anciens imprimeurs) : {\typo texte ici}, la typo etant appliquee uniquement a ce qui se trouve entre les deux balises
%% en LaTeX, on peut faire a la place : \texttypo{texte ici}
%% Par exemple : {\bf texte} et \textbf{texte} auront le meme effet en Latex : mettre texte en gras
%% Quelques exemple :
%% \bf : bold font (gras)
%% \em : emphasis (fait ressortir le texte soit en le mettant en penche si le texte est droit, soit l'inverse)
%% \it : italic (italique)
%% \sl : slanted (penche)
%% \sc : small cap (petites lettres capitales, tres joli)
%% \tt : typewriting text (type machine a ecrire, utile en info pour ecrire des commandes)
%% \sf : typo sans empattement
%% \underline{texte} : souligne le texte
%% \verb! du texte ! : permet d'afficher tel quel (utile pour les underscores, backslash, etc) tout ce qui est ecrit entre les deux points d'exclamation (on peut remplacer les deux points d'exclamation par tout autre symbole, par exemple \verb? texte ? aura le meme effet)
%% \og et \fg : guillements ouvrants et fermants

%%%%%%%%%%%%%%%%%%%%%%%%%%%%%%%%%%%%%%%%%%
\title{
{\normalsize
\vspace*{-1.5cm}
\includegraphics[width=3cm]{eib_logo.jpg}
% 
\hfill
T\textsuperscript{ale}\\
Lycée EIB \'Etoile 2020--2021
\hfill
{\sc Numériques et Sciences Informatiques}
\hrule%
\vspace{5mm}%
}
{\sc Composants intégrés d'un système sur une puce}  %% \sc pour small caps : petites capitales, c'est joli, j'aime bien
}% fin d utilisation de la commande title
\author{Alexandre {\sc Becquaert}}
% \date{} % a remplir et decommenter pour faire apparaitre la date voulue ; decommenter en laissant vide si on ne veut pas de date sur le document

%%%%%%%%%%%%%%%%%%%%%%%%%%%%%%%%%%%%%%%%%%
\begin{document}
%%%%%%%%%%%%%%%%%%%%%%%%%%%%%%%%%%%%%%%%%%
\maketitle       %% execute les commandes de titre (title / author / date 
\tableofcontents %% dresse la table des matieres
\thispagestyle{fancy} %% pour que le style fancy soit applique en 1ere page (par defaut, fancy ne s'applique qu'a partir de la 2eme page)
%%%%%%%%%%%%%%%%%%%%%%%%%%%%%%%%%%%%%%%%%%

%%%%%%%%%%%%%%%%%%%%%%%%%%%%%%%%%%%%%%%%%%
\subsection*{Bulletin officiel} %% l'etoile permet de ne pas numeroter (mais l'entrée n'apparait pas dans la table des matieres : il faut la rajouter manuellement : cf ci-dessous)
\addcontentsline{toc}{section}{Bulletin officiel}    % pour ajouter une ligne à la table des matières (toc pour table of content)
%% ------------------------------------ %%

\begin{center}
\begin{tabular}{|p{5cm}|p{5cm}|p{5cm}|}\hline
{\bf Contenu} & {\bf Capacités attendues} & {\bf Commentaires}\\\hline
Composants intégrés d'un système sur puce.
&
Identifier les principaux composants sur un schéma de circuit et les avantages de leur intégration en termes de vitesse et de consommation. 
&
Le circuit d'un téléphone peut être pris comme un exemple : microprocesseurs, mémoires locales, interfaces radio et filaires, gestion d'énergie, contrôleurs vidéo, accélérateur graphique, réseaux sur puce, etc.
\\\hline
\end{tabular}
\end{center}

%%%%%%%%%%%%%%%%%%%%%%%%%%%%%%%%%%%%%%%%%%
\subsection*{Introduction}
\addcontentsline{toc}{section}{Introduction}    % pour ajouter une ligne à la table des matières (toc pour table of content)
%% ------------------------------------ %%

Brève introduction sur le contexte général et les motivations.

Ici : miniaturisation des composants d'un ordi, et motivations actuelles pour tout mettre sur une puce pour l'informatique \og nomade \fg  (téléphone, tablette, etc), portabilité matérielle, électronique embarquée, etc.

On peut citer ici les sources utilisées dans tout le document, comme par exemple \cite{monlyceenumerique}, \cite{pixees}, \cite{lecluse} et \cite{lewebpedagogique} (que l'on peut aussi citer ponctuellement dans le document si on ne s'en sert juste pour une information ou figure). % \cite{cle de l'entree correspondante dans la bibliographie} : attention il faut compiler plusieurs fois latex pour que toutes les refs s'actualisent)

Ici, le plan s'inspire du plan de \cite{monlyceenumerique}.
%% faire un saut de ligne dans le fichier tex source correspond a un retour a la ligne dans le document final
\\ %% pour un vrai saut de ligne dans le document final, il faut un double backslash


C'est également ici que l'on peut introduire les acronymes importants utilisés tout le long ou une partie du document :

SoC (\og {\em System on Chip}\fg ou \og {\em système sur puce}\fg en français)
\\

Quelques rappels TeX / LaTeX : figure et algorithme (à terme : figure et algo à supprimer de l'intro)
\begin{figure}[!h] %% l'option h! permet de forcer l'endroit de la figure (avec les options t et b : les figures sont forcees en haut de page (t comme top) ou bas (b comme bottom)
\begin{center}
  \includegraphics[height=2.5cm]{by-nc-sa} %% remplacer by-nc-sa par le nom du fichier image a inclure (extension de preference en png, jpeg, ou pdf si on utilise pdfLaTeX) ; le fichier peut etre dans un sous-repertoire, mais c'est plus simple (s'il y a moins d'une dizaine d'images) de tout laisser dans le meme repertoire
  %% height, width ou scale permettent de controler la taille de l'image
  \caption{Titre de la figure : logo de la licence libre CC BY-NC-SA.}
  \label{fig1}
\end{center}
\end{figure}

\begin{center}
\begin{algorithm}[h!]
%% on redefinit les intitules des mots-cles d'algo
\SetKwInOut{Input}{Entr\'ee}
\SetKwInOut{Output}{Sortie}
\SetKwIF{If}{ElseIf}{Else}{Si}{Alors}{Sinon Si}{Sinon}{FinSi}
\SetKwFor{For}{Pour}{Faire}{FinPour}
\SetKwFor{While}{TantQue}{Faire}{FinTantque}
\SetKw{Return}{Retourner}
%
\Input{entrée de l'algo}
\Output{sortie de l'algo}
\BlankLine
\tcp{un commentaire à la C++}
une variable $\leftarrow$ une valeur\\ %% exemple d'affectation ; il faut finir les instructions par un saut de ligne : \\
Rédiger\_rapport(param1 = $n$ élèves)\\ %% exemple d'appel de fonction
\While{une condition avec un {\bf ou} en gras}{  %% TantQue
  faire\_un\_truc\\
  faire\_un\_autre\_truc\\
}
\eIf{une condition}%% \eIf  Si-Sinon : \eIf{condition}{bloc du si}{bloc du sinon}
  {bloc du si}
  {bloc du sinon}
\Return{un truc si l'algo a bien taffé}
\caption{Nom de l'algorithme.}
\label{algo1}
\end{algorithm}
\end{center}

On peut faire référence dans le texte (et il le faut) à la figure ou à l'algorithme avec Fig. \verb!ref{fig1}! ou Algo. \verb!\ref{algo1}! (fig1 et algo1 étant les clés/label associés), ce qui donne : Fig. \ref{fig1} et Algo. \ref{algo1}.

%%%%%%%%%%%%%%%%%%%%%%%%%%%%%%%%%%%%%%%%%%
\section{Composants et architecture générale d'un ordinateur}
%%%%%%%%%%%%%%%%%%%%%%%%%%%%%%%%%%%%%%%%%%

\begin{itemize}
\item Notion d'ordinateur
\item Notion de composant
\item Présentation très grossière des composants et liens entre eux dans un ordinateur classique (où tous les composants sont \og séparés\fg)
\end{itemize}


\begin{figure}[!h]
\begin{center}
%   \includegraphics[height=2.5cm]{}
  \caption{Schéma / photo d'un ordinateur classique.}
  \label{etiquette_de_la_figure_pour_y_faire_reference_plus_tard}
\end{center}
\end{figure}

%%%%%%%%%%%%%%%%%%%%%%%%%%%%%%%%%%%%%%%%%%
\section{Système sur puce}
%%%%%%%%%%%%%%%%%%%%%%%%%%%%%%%%%%%%%%%%%%

Une petite phrase d'introduction : en gros c'est quoi un Soc.

%%%%%%%%%%%%%%%%%%%%%%%%%%%%%%%%%%%%%%%%%%
\subsection{Généralités}
%% ------------------------------------ %%

Définition et principe d'un système sur puce.

Différence avec un système informatique / ordinateur classique.

%%%%%%%%%%%%%%%%%%%%%%%%%%%%%%%%%%%%%%%%%%
\subsection{Présentation et rôle des composants}
%% ------------------------------------ %%

\begin{description}
\item [CPU :]

\item [GPU : ]

\item [ainsi de suite]

\end{description}


\begin{figure}[!h]
\begin{center}
%   \includegraphics[height=2.5cm]{}
  \caption{Schéma d'un système sur puce.}
  \label{etiquette_de_la_figure_pour_y_faire_reference_plus_tard}
\end{center}
\end{figure}

%%%%%%%%%%%%%%%%%%%%%%%%%%%%%%%%%%%%%%%%%%
\section{Types et utilisation des systèmes sur puces}
%%%%%%%%%%%%%%%%%%%%%%%%%%%%%%%%%%%%%%%%%%

Là aussi mini phrase d'intro de la partie.

%%%%%%%%%%%%%%%%%%%%%%%%%%%%%%%%%%%%%%%%%%
\subsection{Les principaux types de systèmes sur puces}
%% ------------------------------------ %%

\begin{itemize}
\item autour d'un microcontrôleur :

\item autour d'un microprocesseur :

\item dédié à une tâche spécifique :
\end{itemize}

%%%%%%%%%%%%%%%%%%%%%%%%%%%%%%%%%%%%%%%%%%
\subsection{Utilisation des systèmes sur puces}
%% ------------------------------------ %%

Exemples d'utilisation et quelques photos.

\begin{figure}[!h]
\begin{center}
%   \includegraphics[height=2.5cm]{}
  \caption{Photo de système sur puce (avec référence).}
  \label{etiquette_de_la_figure_pour_y_faire_reference_plus_tard}
\end{center}
\end{figure}

%%%%%%%%%%%%%%%%%%%%%%%%%%%%%%%%%%%%%%%%%%
\begin{thebibliography}{00}
\addcontentsline{toc}{section}{Références}  %% les references c'est important, on les rajoute a la table des matieres
%% ------------------------------------ %%

\bibitem{pixees}
  {\em Site Pixees}, D. Roche.
  \url{https://pixees.fr/informatiquelycee/n_site/nsi_term_archi_soc.html}

\bibitem{monlyceenumerique}
  {\em Site Mon Lycée Numérique}, T. Lourdet, J. Monteillet, J-C. Gérard \& P. Thérèse.
  \url{http://monlyceenumerique.fr/nsi_terminale/arse/a1_systeme_\%20sur\%20_puce.html}

\bibitem{lecluse}
  {\em Site de NSI du lycée Salvador Allende (Caen)}, O. Lecluse.
  \url{https://www.lecluse.fr/nsi/NSI_T/archi/soc/}

% \bibitem{mathsduyeti}  %% RIEN SUR LES SYSTMES SUR PUCES POUR LE MOMENT
%  {\em Site Les Maths du Yeti}, \og Mike le Yeti\fg (lycée Charles Péguy).
%  \url{}
  
\bibitem{lewebpedagogique}
  {\em Site Le Web Pédagogique}.
  \url{https://lewebpedagogique.com/dlaporte/category/nsi-1ere/}

\bibitem{wikipedia}
  {\em Wikipédia}, article \og System on a chip\fg.
  \url{https://en.wikipedia.org/wiki/System_on_a_chip}

\bibitem{deusexsilicium}
  {\em Chaîne youtube Deus Ex Silicium}.
  \url{https://www.youtube.com/watch?v=ee-LhNZPZ1U&feature=youtu.be}

\end{thebibliography}

%%%%%%%%%%%%%%%%%%%%%%%%%%%%%%%%%%%%%%%%%%
\end{document}
%%%%%%%%%%%%%%%%%%%%%%%%%%%%%%%%%%%%%%%%%%
