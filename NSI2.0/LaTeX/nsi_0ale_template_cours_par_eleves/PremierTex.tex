% Tout ce qui est mis derrière un « % » n'est pas vu par LaTeX.
% Ce sont des commentaires. Ils permettent aussi de structurer le code LaTeX pour le rendre plus clair.

%%%%%%%%%%%%%%%%%%%%%%%%%%%%%%%%%%%%%%%%%%
\documentclass[a4paper,11pt]{article}
%%%%%%%%%%%%%%%%%%%%%%%%%%%%%%%%%%%%%%%%%%

% On commence toujours par préciser la classe du document, ici « article », et des options entre crochets.
%
% Plein de classes possibles : article, book, amsart, amsbook, slides, etc
% Options possibles : 10pt, 11pt, 12pt (taille de la fonte)
%                     oneside, twoside (recto simple, recto-verso)
%                     draft, final (stade de développement)

% Ensuite, on inclut des librairies. Ca donne accès à plein de trucs supers !
% Il en existe un nombre incroyable, mais il ne faut pas en abuser.
% Trop en mettre : 
%    -> rend le code moins lisible, 
%    -> crée des bugs à la compilation (incompatibilités), 
%    -> et peut énerver vos co-auteurs (librairies non standards, versions différentes, etc).

% Ces trois-là sont indispensables :
\usepackage[utf8]{inputenc}    % Pour que LaTeX comprenne les accents.
\usepackage[T1]{fontenc}       % Une autre police de caractères
\usepackage[french]{babel}     % Traitement du texte adapté aux règles typographiques
                               % de la langue donnée en option (e.g., pour l'espacement
                               % après les ponctuations

% Il existe beaucoup de police de caractères à disposition :                               
\usepackage{times}             % Police de caractères

% D'autres peuvent être ajoutées suivant les besoins :
\usepackage{graphicx}                  % pour inclure des figures 
\usepackage{amsmath, amsthm, amssymb}  % Pour faire de jolies mathématiques, 
                                       % avec plein de symboles supplémentaires.
\usepackage{url}                       % pour inclure des urls
\usepackage{hyperref}                  % pour les liens (internes et externes au document)
\usepackage[french,ruled,vlined]{algorithm2e}% pour présenter de jolis algos

% \usepackage[left=3cm,right=3cm,top=3cm,bottom=3cm]{geometry}
                                       % pour bidouiller les marges

% En fait, il existe des librairies pour tout et n'importe quoi en LaTeX : 
% \usepackage{simpsons}                % essayez les commandes \Bart, \Homer,... avec ce package

% D'autres commandes dans l'entête : LaTeX est un langage de programmation !
% On peut faire plein de choses...
\sloppy                                % Ne pas faire déborder les lignes dans la marge
\newcommand{\eps}{\varepsilon}         % Définition d'une nouvelle commande personnalisée (définie dans tout le document)
\newcommand{\ps}[2]{\left<#1,#2\right>}% Définition d'une autre nouvelle commande
% En général, on définit une commande perso, ou une macro, par :
% \newcommand{\nomCommande}{définition commande}
% Par exemple :
% \newcommand{\ada}{\texttt{Ada}\xspace}
% La commande \ada dans le fichier aura pour conséquence, après compilation, d'insérer le texte Ada écrit en écriture tt (police type machine à écrire) dans le pdf final (\xspace rajoute un espace derrière)

% Un petit nombre de champs à compléter avant de commencer le document proprement dit :
\title{Débuter en \LaTeX\footnote{d'après un document de Lucie Martinet et Damien Stehl\'e.}} % \footnote permet de rajouter une note en bas de page, numerotée automatiquement
\author{Maxime \textsc{Senot}}
% \date{} 
% On peut mettre, la date qu'on veut.
% « \today » ou aucun argument permet d'avoir la date d'aujourd'hui (jour de compilation du document).

% Et voilà, on peut commencer le document : 

%%%%%%%%%%%%%%%%%%%%%%%%%%%%%%%%%%%%%%%%%%
\begin{document}
%%%%%%%%%%%%%%%%%%%%%%%%%%%%%%%%%%%%%%%%%%

\maketitle % pour avoir le titre, l'auteur, la date : tout se fait tout seul

\tableofcontents % pour avoir la table des matières
                 % peut être commenté au besoin.

% Idem: le résumé (ou abstract) peut aussi être commenté au besoin.
\begin{abstract}
\LaTeX{} est un langage de programmation pour faire de la mise en page de texte, 
que ce soit un article sur une ou deux colonnes, un livre, des transparents (rendez-vous en avril pour la soutenance !)\ldots 
% \ldots donne les pointillés, avec l'espacement adéquat

% Un retour à la ligne dans le code source \LaTeX{} n'a aucune conséquence dans le fichier généré
% En revanche, cela permet de mieux structurer le contenu dans le fichier source.tex (en évitant par exemple les phrases qui dépassent de l'écran)
% Pour obtenir un retour à la ligne dans le document final, on peut soit utiliser \\, soit faire un saut de ligne (et non un simple retour à la ligne) dans le fichier source

\noindent \LaTeX{} n'est pas un éditeur WYSIWYG (contrairement à Word : \emph{What You See Is What You Get}): 
comme pour tout langage de programmation, les fichiers sources doivent être compilés.
% \noindent permet de ne pas effectuer de retrait en début de paragraphe.

Dans \LaTeX{}, tout est automatisé (entête, numérotation des parties, de la biblio, des notes de bas de  page, etc). 
Le langage est très riche (formules mathématiques, bibliographie, références, table des matières, figures, pseudo-codes, etc). 
La base d'utilisateurs est gigantesque (documentations, faq, librairies). 
% Et de toute fa\c{c}on, vous n'avez pas le choix: c'est le standard commun aux scientifiques!
\end{abstract}

% Le document peut être découpé en parties (assez rare, mais peut arriver pour un gros rapport ou une thèse), sections, sous-sections, sous-sous-sections, etc.
% \part{Titre}                % Commencer une partie...

%%%%%%%%%%%%%%%%%%%%%%%%%%%%%%%%%%%%%%%%%%
\section{L'environnement de travail}   
%%%%%%%%%%%%%%%%%%%%%%%%%%%%%%%%%%%%%%%%%%

% Commencer une section
Un document \LaTeX{} a une extension \texttt{.tex}. 
% La commande \texttt permet de changer le style de police, ici en typewriter typeface.
% Autres possibilités :
%  \emph{texte}    (en italique)
%  \textbf{texte}  (en gras)
%  \textsc{texte}  (en petite capitale)

% Les commandes commencent toutes par un « \ »
% On a désormais 4 caractères réservés : %, \, { et }

C'est le fichier \texttt{.tex} qui va être compilé. 
On peut écrire du \LaTeX{} avec n'importe quel éditeur de texte (gedit, emacs,...). 
Pour compiler, il faut soit utiliser un terminal soit un éditeur qui  permet de compiler avec des raccourcis.

%% ------------------------------------ %%
\subsection{Compiler avec un terminal}
%% ------------------------------------ %%

Il faut d'abord se placer dans le répertoire où se trouve votre \texttt{.tex}  (comme d'habitude)  et lancer l'une des commandes suivantes:

\begin{verbatim}
latex PremierTex.tex
pdflatex PremierTex.tex
\end{verbatim}

% verbatim permet d'insérer du code facilement.
% Cet environnement permet d'afficher exactement ce que l'on voit dans l'éditeur de texte
% Tout ce qui est entre \begin{verbatim} et \end{verbatim} sera directement affiché (sans être compilé).

La première commande permet d'obtenir un \texttt{.dvi}. 
On pourra ensuite l'exporter en \texttt{.pdf}  avec la  commande \texttt{dvipdf PremierTex.dvi}), 
ou en \texttt{.ps} avec la commande \texttt{dvips -o PremierTex.ps PremierTex.dvi}.

La seconde commande est plus pratique, car elle permet d'avoir un \texttt{.pdf} directement.
Mais il peut y avoir des incompatibilités en particulier dans le cas d'une insertion de figures dans d'autres formats que \texttt{.pdf}.

Pour que la table des matières (qui s'obtient avec la commande \texttt{$\backslash$tableofcontents}, 
% Ben oui, le caractère \ étant réservé, il faut ruser pour pouvoir écire \
placée dans le corps du document \LaTeX{}) soit à jour, il faut compiler deux fois de suite. 
C'est aussi le cas pour les références, ou pour la bibliographie.

%% ------------------------------------ %%
\subsection{Compiler avec un éditeur plus ou moins spécifique}
%% ------------------------------------ %%

Ce document a été initialement compilé avec TeXworks (disponible sous windows). 
Dans la plupart des éditeurs spécifiques à \LaTeX{}, il existe un bouton permettant 
d'effectuer la compilation du fichier \texttt{.tex} directement en fichier \texttt{.pdf},
De nombreux autres éditeurs de textes spécifiques à \LaTeX{} existent : Kile, TexMaker, TeXnicCenter (Windows)\ldots

%%%%%%%%%%%%%%%%%%%%%%%%%%%%%%%%%%%%%%%%%%
\section{Commencer un document}
%%%%%%%%%%%%%%%%%%%%%%%%%%%%%%%%%%%%%%%%%%

Comme pour {\tt Python}, {\tt C}, {\tt Java}, ou tout autre langage de programmation, un programme \LaTeX{} commence par des entêtes et une liste de packages à importer.

L'une des plus courtes formes se présente de la manière suivante:

\begin{verbatim}
\documentclass[10pt]{TypeDeDocument}
\usepackage[utf8]{inputenc}
\usepackage[frenchb]{babel}
\usepackage[T1]{fontenc}

\date{}

\begin{document}
\section{Titre de la première section}
\section{Titre de la deuxième section}
\end{document}
\end{verbatim}


Avant \texttt{$\backslash$begin\{document\}}, 
% On ruse maintenant pour obtenir les caractères réservés { et }.
nous pouvons aussi ajouter nos propres commandes en les déclarant préalablement.

Ce qui arrive après la commande \texttt{$\backslash$end\{document\}} n'est pas pris  en compte lors de la compilation.

%%%%%%%%%%%%%%%%%%%%%%%%%%%%%%%%%%%%%%%%%%
\section{Mise en forme du document}
%%%%%%%%%%%%%%%%%%%%%%%%%%%%%%%%%%%%%%%%%%

%% ------------------------------------ %%
\subsection{Mise en forme générale du document}
%% ------------------------------------ %%

Pour faire des sections numérotées avec un titre, on utilise la commande suivante:
\begin{verbatim}
\section{Titre}
\end{verbatim}

Pour faire des sous-sections:
\begin{verbatim}
\subsection{Sous-Titre}
\end{verbatim}

Pour faire des sous-sous-sections:
\begin{verbatim}
\subsubsection{Sous-Sous-Titre}
\end{verbatim}

Et ça s'arrête là. Pour supprimer la numérotation des sections, 
il faut ajouter une * comme suit : 
\begin{verbatim}
\section*{Titre}
\subsection*{Sous-Titre}
\subsubsection*{Sous-Sous-Titre}
\end{verbatim}

L'indentation se fait automatiquement à chaque nouveau paragraphe. 
Un nouveau paragraphe est créé quand il y a un saut de ligne dans le code \LaTeX{}. 
On peut également utiliser les commandes suivantes :
\begin{verbatim}
\\      %pour passer à la ligne suivante
%ou
\par    %pour commencer un nouveau paragraphe
\end{verbatim}

%% ------------------------------------ %%
\subsection{Les listes}
%% ------------------------------------ %%

Il est possible de faire des listes, des énumérations, et même de changer les numéros et la forme des  « tirets ». Par exemple~:

Ma liste de courses:
\begin{itemize}
\item Jus de fruits
\item[$\Rightarrow$] Viande
\item [$\bullet$] Une nouvelle classe de L1
\end{itemize}

\bigskip
% Pour créer des espaces. Il existe aussi \medskip et \smallskip

Ou par ordre de priorité:
\begin{enumerate}
\item Une nouvelle classe de L1
\item Jus de fruits
\item Viande
\end{enumerate}


\bigskip
% Les commandes \smallskip, \medskip et \bigskip permettent d'insérer des sauts de lignes plus ou moins grands 

\LaTeX{} permet également de faire des tableaux :

\begin{tabular}{|lc|l|}
% l centre le texte à gauche, r à droite, et c au centre
% Les barres verticales créent des séparations verticales dans le tableau
\hline
case 1 & case 4 & case 7\\
\hline
case 2 & c & case 8 \\
case 3 & case 6 & case 9\\
\hline
\end{tabular}

\smallskip
On peut aussi centrer du texte, ou n'importe quoi. Par exemple:
\begin{center}
\begin{tabular}{|lr|l|}
% l centre le texte à gauche, r à droite, et c au centre
% Les barres verticales créent des séparations verticales dans le tableau
\hline
case 1 & case 4 & case 7\\
\hline
case 2 & case 5 & case 8\\
case 3 & case 6 & case 9\\
\hline
\end{tabular}
\end{center}

%%%%%%%%%%%%%%%%%%%%%%%%%%%%%%%%%%%%%%%%%%
\section{Inclure des images}
\label{se:images}
% Créer un label permet ensuite de faire référence à cette section
%%%%%%%%%%%%%%%%%%%%%%%%%%%%%%%%%%%%%%%%%%

Pour faire cela, vous devez inclure un package qui le permet :
\begin{verbatim}
\usepackage{graphicx}
\end{verbatim}

A l'endroit où vous voulez inclure votre image, vous devez utiliser la commande suivante :
\begin{verbatim}
\includegraphics{nom_de_la_figure}
\end{verbatim}

Par exemple:
\includegraphics[scale=.5]{by-nc-sa}

Bon, c'est pas super\ldots Pour arranger un peu l'affichage, il faut jouer avec les options, et tester!

En général, lorsqu'on ajoute une image au sein d'un texte, on aime bien pouvoir lui donner une légende et y faire référence. 
Pour cela, il existe un environnement flottant, nommé \emph{figure}. Un environnement flottant signifie, comme son nom l'indique, qu'il peut être déplacé. 
En particulier, \LaTeX{} placera cet environnement à l'endroit qu'il juge le plus adéquat, souvent, à l'endroit qui permet de minimiser la taille du document.
\begin{verbatim}
\begin{figure}[!h]
  \includegraphics{nom_de_la_figure}
  \caption{La légende que vous voulez.}
\end{figure}
\end{verbatim}

L'option \texttt{[!h]} permet de forcer \LaTeX{} à placer notre figure à cet endroit du document et pas ailleurs. 
Malgré cette instruction, \LaTeX{} n'obéit pas toujours, mais c'est rare.

L'instruction \texttt{$\backslash$caption\{La légende que vous voulez\}} permet d'ajouter  une légende à la figure.

Vous pouvez aussi ranger vos figures dans un répertoire séparé, par exemple un sous-répertoire. 
Cependant, vous devrez veiller à écrire le bon chemin.

Voici ce que l'on obtient avec la figure précédente, avec quelques options:
\begin{figure}[!h]
  \begin{center}
  \includegraphics[height=2.5cm]{by-nc-sa}
  % Les options de largeur (width), rotation (angle), échelle (scale) sont également disponibles 
  \caption{Logo de la licence libre CC BY-NC-SA.}
  % Je crée un label pour la figure (attention à bien déclarer le label APRES la légende).
  \label{fig1}
  \end{center}
\end{figure}

%%%%%%%%%%%%%%%%%%%%%%%%%%%%%%%%%%%%%%%%%%
\section{\'Ecrire un algorithme}
\label{se:algo}
%%%%%%%%%%%%%%%%%%%%%%%%%%%%%%%%%%%%%%%%%%

Il existe différents packages pour présenter des algorithmes en \LaTeX{} : \texttt{algorithm2e}, \texttt{algorithmics}\ldots
L'algorithme \og Faire un rapport \LaTeX{} \fg est un exemple d'algorithme obtenu grâce au package \texttt{algorithm2e}.
% \og et \fg sont les guillemets respectivement ouvrant et fermant.
\begin{algorithm}
\SetKwInOut{Input}{input}
\SetKwInOut{Output}{output}
\Input{un rapport de projet à écrire.}
\Output{le rapport rédigé en \LaTeX{}.}
\BlankLine
\tcp{voir la doc d'algorithme2e pour définir des mots-clés et instructions en français}
$n \leftarrow$ nombre de personnes dans l'équipe du projet \\
Rédiger\_rapport(param1 = $n$ élèves ; param2 = consignes) \\
\SetKw{Ou}{ou}
Compiler\_rapport \\
\While{(errors) \Ou (warnings)}{
  Lire\_messages \\
  Corriger\_erreurs \\
  Compiler\_rapport \\
}
\eIf{Accord(param\_opt = toute l'équipe) {\bf or} date=deadline}
  {Rendre\_rapport}
  {Effectuer\_retouches }
\caption{Faire un rapport en \LaTeX{}.}
\end{algorithm}

%%%%%%%%%%%%%%%%%%%%%%%%%%%%%%%%%%%%%%%%%%
\section{Labels, références, notes de bas de page}
%%%%%%%%%%%%%%%%%%%%%%%%%%%%%%%%%%%%%%%%%%

Tout cela peut se faire automatiquement, sans avoir à chercher les numéros qui vont bien manuellement. 
Les commandes importantes pour cela sont:
\begin{verbatim}
\label{}     % pour créer un label
\ref{}       % pour lui faire référence
\cite{}      % idem \ref, mais pour la biblio
\footnote{}  % note de bas de page
\end{verbatim}
Comme j'ai créé des références dans mon fichier source, je peux maintenant parler de la Figure~\ref{fig1} de la Section~\ref{se:images}. 
Je peux également faire référence à l'entrée~\cite{lamport94} de la bibliographie.
\footnote{Ne pas abuser des notes de bas de page}
\footnote{Qu'est-ce que je viens juste de dire?}
% Le caractère ~ fournit un espace insécable
% Le package hyperref fournit un raccourci direct entre la ref (ou le cite) et le label. 
% En bidouillant les options de hyperref, on peut mettre ce lien en couleurs, en boites, le souligner... 

%%%%%%%%%%%%%%%%%%%%%%%%%%%%%%%%%%%%%%%%%%
\section{Les maths}
%%%%%%%%%%%%%%%%%%%%%%%%%%%%%%%%%%%%%%%%%%

Pour faire des maths, on utilise un environnement spécial.
On a le choix entre deux environnements possibles:
\begin{verbatim}
$mon expression mathématiques$
\end{verbatim} 
ou 
\begin{verbatim}
\[mon expression mathématiques\]
\end{verbatim}
La première version permet d'écrire des maths au sein d'une ligne de texte sans revenir à la ligne. 
Un petit exemple valant mieux qu'un long discours, on peut écrire ceci: 
$\displaystyle\sum_{i=0}^{max} i^2$, avec la commande 
\texttt{$\$\backslash displaystyle\backslash$sum\_\{i=0\}\^{ }\{max\} i\^{ }2$\$$}.
% La commande \displaystyle permet d'avoir une jolie mise en page. Mais elle augmente l'écart entre les lignes.

La deuxième crée un environnement qui sera séparé du texte par un retour à la ligne et des sauts de lignes. 
Par exemple, l'expression \texttt{$\backslash$[$\backslash$sum$\_\{i=0\}^{ }n 
\backslash$prod$\_\{$j=i$\}^{ }\{$m$\} $k$\_\{\mathtt{ij}\}^{ }2 \backslash$]} donnera:

\[ \sum_{i=0}^n \prod_{j=i}^{m} k_{ij}^2 \]

On peut aussi créer des commandes \LaTeX{} pour les formules que l'on utilise souvent, et pour rendre le fichier source plus lisible. 
Par exemple, avec les commandes créées au début du source \texttt{.tex} de ce \texttt{.pdf}: $\eps$ et~$\ps{\vec{b}_1}{\vec{b}_2}$.

\newpage
% La commande \newpage permet de forcer une nouvelle page

%%%%%%%%%%%%%%%%%%%%%%%%%%%%%%%%%%%%%%%%%%
\section{Ressources}
%%%%%%%%%%%%%%%%%%%%%%%%%%%%%%%%%%%%%%%%%%

Quelques liens très utiles:\\

\url{www.tuteurs.ens.fr/logiciels/latex}\\

\url{http://melusine.eu.org/syracuse/doc/faq-tex-french/faq-tex-french.html}\\

\url{http://detexify.kirelabs.org/classify.html}\\

\url{www.ctan.org/tex-archive/info/symbols/comprehensive/symbols-a4.pdf}

% Avant de finir, une petite bibliographie :

%%%%%%%%%%%%%%%%%%%%%%%%%%%%%%%%%%%%%%%%%%
\begin{thebibliography}{00}
%%%%%%%%%%%%%%%%%%%%%%%%%%%%%%%%%%%%%%%%%%

% Le nombre de caractère de l'argument en option spécifie le nombre maximal de caractères des entrées bibliographiques. 
% Deux caractères => au plus 99 entrées

\bibitem{lamport94}
  Leslie Lamport,
  \emph{\LaTeX\hspace{-.2cm}: A Document Preparation System}.
% Eh oui, il y a même une commande \LaTeX
  Addison Wesley, Massachusetts,
  2nd Edition,
  1994.

\bibitem{ctan}
  The Comprehensive TeX Archive Network.
  \url{http://www.ctan.org/}

% Et pour finir, la bible de LaTeX :
\bibitem{knuth86}
  Donald E. Knuth,
  \emph{The \TeX{} book}.  
  Addison Wesley,
  1986.

\end{thebibliography}
% Il existe un autre moyen de gérer la bibliographie : en utilisant BibTeX.
% BibTeX permet de générer automatiquement la partie thebibliography avec les bibitems, à partir d'un fichier .bib, dans lequel on réferencie les entrées.
% (attention à l'ordre compilations : LaTeX, puis BibTeX, puis LaTeX à nouveau.

%%%%%%%%%%%%%%%%%%%%%%%%%%%%%%%%%%%%%%%%%%
\end{document}
%%%%%%%%%%%%%%%%%%%%%%%%%%%%%%%%%%%%%%%%%%

